\documentclass{article}
\title{Vissynth Project Proposal}
\author{Chang \\ Evan \\ Sam \\ Yui}
\begin{document}
\maketitle

Vissynth is a visual synthesizer, receiving input from a webcam.
By changing the input to the camera (e.g. moving around
frantically in front of the camera), a user can change the music
produced by the synthesizer.

To accomplish this, we will use the OpenCV library to handle
talking to the webcam. We intend to have our program save
individual images from the camera at a given rate (e.g. 10 Hz),
and then have our program operate on the images in the appropriate
sequence (first image in time will be read first, followed by
second, third, etc.) instead of attempting to handle live input
from the camera.

For every image saved, the program will do (roughly) the
following:

\begin{enumerate}
  \item Turn the contrast of the image to 100\%, leaving only black and white pixels.
  \item Separate the image into a 10x10 grid, each corresponding to a button on the synthesizer (explained below).
  \item Count up the number of white and black pixels within each square of the grid.
  \item For each square, mark that square ON if there are more white than black pixels, OFF otherwise.
\end{enumerate}

We will then use the ON/OFF sequences as input to the synthesizer
in the following manner. Each column of a grid represents a time
point (i.e. column 1 = time 1, column 2 = time 2, etc.), such that
all of the squares within a given column will be played at the
same time. Each row corresponds to a particular frequency, with
lower rows representing lower tones and higher rows representing
higher tones.

Our group would like to design a program that functions as a
visual synthesizer. We intend to take input from a web camera and
use this input to drive a musical synthesizer. To achieve this, we
will separate the image into a grid, which will roughly correspond
to the grid of a synthesizer: an array of buttons, where columns
represent "time steps", every ON button in each column goes off at
the same time point, and the rows represent frequencies, with
lower rows representing lower tones and higher rows representing
higher tones. The program will decide which squares of the grid
are ON and which are OFF based on the following process:

\begin{enumerate}
  \item turn the contrast all the way up on the image
  \item count up the number of white and black pixels within each square of the grid
  \item for each square, mark that square as ON if there are more white than black pixels, OFF otherwise.
\end{enumerate}

\end{document}