\documentclass{article}
\title{Vissynth Project Proposal}
\author{Chang (cjoon@seas.upenn.edu) \\ Evan (ebensh@seas.upenn.edu) \\ Sam (smessing@sas.upenn.edu) \\ Yui (ksuvee@seas.upenn.edu)}
\begin{document}
\maketitle

\section{Project Description} % (fold)
\label{sec:project_description}

Vissynth is a visual synthesizer, receiving input from a webcam.
By changing the input to the camera (e.g. moving around
frantically in front of the camera), a user can change the music
produced by the synthesizer.

To accomplish this, we will use the OpenCV library to handle
talking to the webcam. We intend to have our program save
individual images from the camera at a given rate (e.g. 10 Hz),
and then have our program operate on the images in the
appropriate sequence (first image in time will be read first,
followed by second, third, etc.) instead of attempting to handle
live input from the camera.

For every image saved, the program will do (roughly) the
following:

\begin{enumerate}
  \item Turn the contrast of the image to 100\%, leaving only black and white pixels.
  \item Separate the image into a 10x10 grid, each corresponding to a button on the synthesizer (explained below).
  \item Count up the number of white and black pixels within each square of the grid.
  \item For each square, mark that square ON if there are more white than black pixels, OFF otherwise.
\end{enumerate}

We will then use the ON/OFF sequences as input to the synthesizer
in the following manner. Each column of the grid represents a
time point (i.e. column 1 = time 1, column 2 = time 2, etc.),
such that all of the squares within a given column will be played
at the same time. Each row corresponds to a particular frequency,
with lower rows representing lower tones and higher rows
representing higher tones.

The program will take a grid based on a image, and step through
each time point (each column) sequentially, playing each tone
corresponding to an ON square at the same time. The rate at which
the program takes photos from the webcam will be proportional to
the amount of time it takes to step through one image. So, for
example, if each image has ten columns and each column plays for
.2 seconds, our program will take a new photo once every 2
seconds, or at a rate of 0.5Hz.

In addition, the program will display the given grid being
operated on at each time step (using ncurses or some other
library), highlighting the column currently being played.

Because of the design of this project, we have several different
modules that individuals in the group can work on, including
talking to the camera, writing to the screen, producing sound,
and processing the actual images. This will enable us to clearly
delineate responsibilities and construct a series of deadlines to
reach.

% section project_description (end)

\section{Feature Lists} % (fold)
\label{sec:feature_lists}

\subsection{Checkpoint} % (fold)
\label{sub:checkpoint}

\begin{enumerate}

\item Talking to the camera: be able to access a webcam, and save
images in an ordered fashion.

\item Drawing to the screen: be able to draw a grid object to the
screen.

\item Making sound: be able to play multiple tones at the same
time.

\end{enumerate}

% subsection checkpoint (end)

\subsection{Final Submission} % (fold)
\label{sub:final_submission}

\begin{enumerate}

\item Operating over images: transform the images into grid
objects for use in the synthesizer.

\item Real time sound processing: be able to step through given
grid objects and play appropriate tones at appropriate times.

\item Real time screen drawing: be able to update the screen in
real time to reflect what the synthesizer is doing.

\end{enumerate}

% subsection final_submission (end)

% section feature_lists (end)

\section{Dates} % (fold)
\label{sub:dates}

For checkpoint review:

\begin{enumerate}
  \item 11/23/2010: 2PM
  \item 11/23/2010: 3PM
  \item 11/23/2010: 4PM
\end{enumerate}

\noindent For final submission:

\begin{enumerate}
  \item 12/15/2010
\end{enumerate}

% subsection dates (end)

\section{Future Directions} % (fold)
\label{sec:future_directions}

\begin{enumerate}
  \item Implement some sort of "beat", to make sure program runs at regular intervals.
\end{enumerate}

% section future_directions (end)

\end{document}